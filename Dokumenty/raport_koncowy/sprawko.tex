\documentclass[a4paper]{article}

\usepackage[utf8]{inputenc}
\usepackage[polish]{babel}
\usepackage{polski}
\usepackage{listings}
\usepackage[T1]{fontenc}
\usepackage[margin=0.9in]{geometry}
\usepackage[usenames,dvipsnames]{xcolor}
\usepackage{lmodern}
\usepackage{pdfpages}
\usepackage{float}
\usepackage{longtable}
\usepackage{graphicx}
\usepackage{pdfpages}
\usepackage{multirow}

\begin{document}

\begin{titlepage}

\newcommand{\HRule}{\rule{\linewidth}{0.5mm}} 

\center 
 
%----------------------------------------------------------------------------------------
%	HEADING SECTIONS
%----------------------------------------------------------------------------------------

\textsc{\LARGE Politechnika Wrocławska}\\[1.0cm] % Name of your university/college
\textsc{\Large Bazy Danych}\\[0.2cm] % Minor heading such as course title
\textsc{\large Raport końcowy}\\[2cm]

%----------------------------------------------------------------------------------------
%	TITLE SECTION
%----------------------------------------------------------------------------------------

\HRule \\[0.4cm]
{ \huge \bfseries Analizator danych pogodowych}\\[0.4cm] % Title of your document
\HRule \\[3cm]
 
%----------------------------------------------------------------------------------------
%	AUTHOR SECTION
%----------------------------------------------------------------------------------------

\begin{minipage}{0.4\textwidth}
\begin{flushleft} \large
\emph{Autorzy:}\\
Aleksandra \textsc{Grzelak}\newline
Dorian \textsc{Janiak}\newline
Marcin \textsc{Ochman}

\end{flushleft}
\end{minipage}
~
\begin{minipage}{0.4\textwidth}
\begin{flushright} \large
\emph{Prowadzący:} \\
dr hab. inż. Grzegorz \textsc{Mzyk}
\end{flushright}
\end{minipage}\\[12cm]

{\large \today}\\[3cm] 


\vfill 

\end{titlepage}

\newpage
\tableofcontents
\listoffigures
\newpage

\section{Opis projektu}
W ramach projektu powstała baza danych zawierająca dane pomiarowe ze stacji pogodowych oraz interfejs graficzny w postaci strony internetowej. Do zrealizowania zadania posłużył serwer \textbf{MySQL} (przetrzymywanie oraz udostępnianie danych), język \textbf{Python} (logika aplikacji) wraz z modułem \textbf{Django} (framework web - strona graficzna oraz zarządzająca bazą). 
\section{Opis funkcjonalności}
Stworzony przez nas analizator realizuje poniżej wymienione funkcje:
\begin{itemize}
	\item \textbf{Wczytywanie danych pogodowych z plików CSV} - plik ma określony format (opisany w punkcie: 7.2). Funkcja uaktywnia się jedynie dla zalogowanych użytkowników aplikacji. Dostępna jest z poziomu panelu sterowania położonego w górnej części strony (''Wczytaj dane''). Wczytuje dane z pliku jednocześnie wpisując je do tabeli \verb Analyzer_danepomiarowe . 
	\item \textbf{Logowanie użytkownika} - logowanie odbywa się z poziomu górnego panelu sterowania (''zaloguj się''). Dane użytkownika domyślnie zapisane są w tabeli \verb auth_user . Jeśli wpisane hasło lub login nie pokryją się z zawartością bazy zostanie wyświetlony monit o niepoprawnym logowaniu. W przeciwnym wypadku logowanie przebiegnie pomyślnie i użytkownik otrzyma dostęp do funkcji wczytywania danych pogodowych. Została zaimplementowana również możliwość wylogowania użytkownika.
	\item \textbf{Rysowanie wykresów} - wykresy rysowane są gdy użytkownik wybierze jedną z opcji związaną z podglądem danych pogodowych. Do rysowania wykorzystywany jest pakiet \textbf{matplotlib}. Dzięki wykorzystaniu tej biblioteki stworzenie wykresu na stronie jest bardzo podobne do generowania wykresów w pakiecie Matlab.
	\item \textbf{Wybór stacji i rodzaju danych pomiarowych} - użytkownik wybiera stację oraz parametr, którego wykres chce wyświetlić. W programie po uruchomieniu funkcji reagującej na naciśnięcie odpowiedniego przycisku zostaje stworzony obiekt klasy Algorithm, który zawiera zestaw danych dla wybranej stacji oraz rodzaju pomiaru. 
	\item \textbf{Usuwanie danych pomiarowych} - jeśli użytkownik jest zalogowany ma możliwość z poziomu zakładki ''podgląd stacji pogodowych'' usunąć dane. 
	\item \textbf{Prognoza} - w programie został zaimplementowany algorytm prognozowania. Został on oparty na modelu \textbf{ARMA}. Algorytm nie jest jednak czystym prognozowaniem ARMA, został on zmodyfikowany. W naszej implementacji opiera się on również na interpolacji oraz wyznaczaniu wzmocnień.
\end{itemize}
\section{Tabele}
W raporcie opisujemy jedynie te tabele, które zostały utworzone bezpośrednio przez nas, ponieważ baza danych zawiera dodatkowo tabele, które tworzone są przez Django w momencie inicjalizacji projektu.
Poniżej zamieszczony został diagram przedstawiający tabele i relacje między nimi:
\begin{figure}[H]
\centering
\includegraphics[width=\linewidth]{diagramEER}
\caption{Diagram tabel}
\label{fig:diagramTabel}
\end{figure}

 Poniższa tabela zawiera pełne zestawienie pól stworzonych tabel.
\begin{longtable}{|p{0.25\linewidth}|p{0.25\linewidth}||p{0.5\linewidth}|}
\hline
\textbf{Nazwa tabeli} & \textbf{Nazwa pola} & \textbf{Opis pola} \tabularnewline \hline \hline
\multirow{5}{*}{Analyzer\_danepomiarowe} & id & klucz główny (automatycznie inkrementowany) \tabularnewline\cline{2-3}
 & wartosc & całkowita część danej pomiarowej \tabularnewline\cline{2-3}
 & rodzaj\_pomiaru\_id & klucz obcy (tabela Analyzer\_rodzajpomiaru) \tabularnewline\cline{2-3}
 & stacja\_id & klucz obcy (tabela Analyzer\_stacja)\tabularnewline\cline{2-3}
 & data & data (godzina jest ignorowana) \tabularnewline\hline
\multirow{2}{*}{Analyzer\_jednostka} 
 & id & klucz główny (automatycznie inkrementowany) \tabularnewline\cline{2-3}
 & nazwa & jednostka pomiaru (np. C lub F dla temperatury) \tabularnewline\hline
\multirow{3}{*}{Analyzer\_rodzajpomiaru}
 & id & klucz główny (automatycznie inkrementowany) \tabularnewline\cline{2-3}
 & nazwa & nazwa rodzaju pomiaru, która będzie wykorzystywana do wyboru danych pomiarowych w aplikacji internetowej \tabularnewline\cline{2-3}
 & jednostka\_id & klucz obcy (tabela Analyzer\_jednostka \tabularnewline\hline
\multirow{2}{*}{Analyzer\_stacja}
 & id & klucz główny (automatycznie inkrementowany) \tabularnewline\cline{2-3}
 & nazwa & nazwa stacji, która będzie wykorzystywana do wyboru danych pomiarowych w aplikacji internetowej \tabularnewline\hline
\end{longtable}

\subsection{Opis tabel}
Tabele zostały specjalnie podzielone w taki sposób, aby móc w przyszłości łatwo dodawać kolejne wpisy. Poniżej krótko opisujemy stworzone tabele:
\begin{itemize}
	\item \textbf{Analyzer\_danepomiarowe} - w tabeli tej znajduje się pełne zestawienie danych pomiarowych. Dzięki wykorzystaniu kluczów obcych \textit{ rodzaj\_pomiaru\_id} oraz \textit{stacja\_id} można potem z tabeli filtrować potrzebne dane. Ponieważ w projekcie skupiliśmy się na pomiarach temperatury pole \textit{wartosc} reprezentowane jest przez liczbę całkowitą INT ze znakiem. Kolumna \textit{data} musi zawierać datę wykonania pomiaru, może również być w niej zapisana godzina, aczkolwiek zostanie ona w naszym programie zignorowana. W tej sytuacji można zauważyć, że domyślnie w naszej tabeli znajdują się daty w stylu \verb 2013-12-01  \verb 00:00:00  - godzina wynosi 0. Założyliśmy, że nie będą potrzebne nam pomiary z rozdzielczością godzinową.
	\item \textbf{Analyzer\_jednostka} - jednostka pomiaru została specjalnie wydzielona do osobnej tabeli, ponieważ w bazie może zdarzyć się sytuacja gdy przykładowo stworzony zostanie rodzaj pomiaru TMIN (temperatura minimalna) oraz TMAX (temperatura maksymalna), które mimo, że są różnymi rodzajami pomiaru, posiadają taką samą jednostkę - stopnie Celsjusza. Jednostka może zostać zapisana w maksymalnie 20 znakach, co oczywiście w takim wypadku jest bardzo dużym zapasem. Ponieważ jednostek w przypadku pogody nie będzie dużo nie należy przejmować się, że taka długość napisu może zająć zbyt dużo pamięci w bazie. 
	\item \textbf{Analyzer\_rodzajpomiaru} - rodzajami pomiaru mogą być np. temperatura, ciśnienie, szybkość wiatru itd. Nazwa, która zostanie wpisana w kolumnie \textit{nazwa} będzie widoczna później w aplikacji web. Z tego powodu można wpisać w jej ramach do 20 znaków. Rodzaj pomiaru powiązany jest kluczem obcym z jednostką. Później jest wykorzystywany przy wczytywaniu danych z pliku CSV.
	\item \textbf{Analyzer\_stacja} - tabela przechowuje poza kluczem głównym jedynie nazwy poszczególnych stacji. Na nazwę przyjęliśmy typ VARCHAR(30), czyli może składać się maksymalnie z 30 znaków. Nazwa stacji jest widoczna w aplikacji webowej oraz jest istotna przy wczytywaniu danych z pliku CSV.
\end{itemize}



\section{Użytkownicy}
Ponieważ dane pogodowe pochodzą jedynie od stacji pogodowych, a więc tylko one powinny mieć uprawnienia do wprowadzania pomiarów. Z tego powodu użytkownicy korzystający z bazy danych zostali podzieleni na dwie grupy:
\begin{itemize}
	\item \textbf{użytkownicy anonimowi} - to tacy użytkownicy, którzy mają dostęp jedynie do przeglądania danych pogodowych.
	\item \textbf{użytkownicy zarejestrowani} - to tacy użytkownicy, którzy mają również możliwość dodawania danych pogodowych oraz ich usuwania. Ponieważ każda stacja musi zostać zweryfikowana (aby nie publikowała fałszywych danych) przez administratorów bazy danych, w aplikacji internetowej nie zostało dodane pole rejestracji użytkowników. Zarejestrowani mogą być bezpośrednio przez administratorów poprzez ręczny wpis ich danych potrzebnych do logowania.  
\end{itemize}
Poniżej zamieszczona została tabela zestawiająca funkcję i możliwość korzystania z niej przez poszczególnych użytkowników. Pola oznaczone znakiem X oznaczają, że dany typ użytkownika posiada uprawnienia do wykonywania czynności; w przeciwnym wypadku użytkownik nie ma takiej możliwości.

\begin{longtable}{|p{0.5\linewidth}|p{0.25\linewidth}||p{0.25\linewidth}|}
\hline
\textbf{Funkcja} & \textbf{Anonimowy} & \textbf{Zarejestrowany} \tabularnewline \hline \hline
Wczytanie danych z pliku CSV & - & X \tabularnewline\hline
Usunięcie danych pogodowych & - & X \tabularnewline\hline
Wybór stacji i rodzaju danych pomiarowych & X & X \tabularnewline\hline
Podgląd wykresów pogodowych & X & X \tabularnewline\hline
Podgląd prognozy pogody & X & X \tabularnewline\hline
\end{longtable}


\section{Sprawozdanie z implementacji i dokumentacji}
Projekt był realizowany w zespole 3 osobowym. Aby zapewnić dobry poziom komunikacji między członkami zespołu organizowane były 3 spotkania, w trakcie których omówione zostały kolejno:
\begin{itemize}
	\item Założenia odnośnie tego czym ma być i jakie funkcje ma realizować aplikaca bazo-danowa.
	\item Założenia odnośnie organizacji bazy danych (powiązania między tabelami oraz wybór pakietu django).
	\item Podsumowanie osiągniętych wyników.
\end{itemize}
Poza tym zespół kontaktował się ze sobą na portalu facebook.com na bierząco informując pozostałych członków nt. postępów. Kod został uwspólniony przy użyciu serwisu github.com. 
\newline
Najpierw powstała testowa baza danych, która była wymagana do uruchomienia projektu. Logika została zrealizowana na bazie przykładów dostępnych w internecie. Następnie część logiczna (python i django) została połączona z bazą danych MySQL (odpowiednie modyfikacje pliku settings.py). Aby móc zobaczyć wyniki - działającą stronę należało utworzyć dodatkowo plik analyzer\_main.html oraz w plikach urls.py i views.py powiązać go. W ten sposób udało się uruchomić pierwszy projekt z Django.
\newline
Dostarczając raport nr 1 skorzystaliśmy z gotowej aplikacji, która na podstawie istniejącej już bazy danych generuje diagram tabel. 
Następnie, aby zintegrować bazę danych z logiką i móc korzystać z zawartości bazy wyedytowaliśmy plik models.py, w którym stworzone zostały odpowiednie klasy (modele - odpowiednik tabeli) i dodane do nich odpowiednie pola (poszczególne kolumny tabeli). Obiekt takiej klasy odpowiada jednemu wierszowi tabeli. 
\newline
Prace programistyczne były przeplecione analizą algorytmu ARMA. Po znalezieniu odpowiednich przykładów, wykorzystujących pakiet statsmodels, rozpoczęliśmy w wydzielonym od projektu katalogu testy i próbę stworzenia prognozy pogody. Po kilku testach okazało się, że model ARMA nie był wystarczający do otrzymania sensownej prognozy, a więc postanowiliśmy zaimplementować własny algorytm oparty na modelu ARMA, interpolacji i pewnym wyliczaniu wzmocnień.
\newline
Po skończeniu prac nad prognozą została ona włączona do aplikacji oraz dołączono możliwość logowania użytkownika.
\newline
Na tym etapie zostały przedstawione wyniki w trakcie II spotkania kontrolnego. 
W maju natomiast zostały wprowadzone poprawki w bazie danych oraz stworzony raport końcowy.

\section{Interfejs}
Interfejs graficzny jest tworzony przez Django. Django na podstawie szablonów w postaci plików html oraz danych otrzymanych z funkcji napisanych w python'ie przygotowuje kompletny podgląd strony. 
Strona główna (Rysunek \ref{fig:stronaStartowa}) składa się z 3 dużych przycisków animowanych przy użyciu javascript, które doprowadzają do kolejnych zakładek. W górnej części strony znajdują się przyciski (''Indeks'', ''Podgląd stacji pogodowych'', ''Prognoza'', \textit{''Wczytaj dane''}, ''Autorzy''). W przypadku przycisku ''Wczytaj dane'' nie pojawi się on jeśli użytkownik jest niezalogowany. \newline 
\begin{figure}[p]
	\centering
	\includegraphics[width=\textheight, angle=90]{000}
	\caption{Strona startowa}
	\label{fig:stronaStartowa}
	\end{figure}

Po kliknięciu na zakładki ''Podgląd stacji pogodowych'' oraz ''Prognoza'' otwiera się bardzo podobna podstrona (Rysunek \ref{fig:podgladStacji}), która służy do wyboru rodzaju pomiaru oraz stacji, dla której dane mają zostać wyrysowane. W przypadku zalogowanych użytkowników pojawia się również możliwość skasowania danych w postaci przycisku ''delete''. 
\newline
\begin{figure}[p]
	\centering
	\includegraphics[width=\linewidth]{001}
	\caption{Podgląd stacji pogodowych - użytkownik zalogowany}
	\label{fig:podgladStacji}
	\end{figure}

Wykresy rysowane są przy użyciu biblioteki matplotlib (Rysunek \ref{fig:wykres}). 
\newline
\begin{figure}[p]
	\centering
	\includegraphics[width=\linewidth]{002}
	\caption{Wyrysowany wykres}
	\label{fig:wykres}
	\end{figure}
\subsection{Panel administracyjny}
Django automatycznie dostarcza panel administracyjny, który pozwala na wygodne zarządzanie danymi w bazie danych. Aby się do niego dostać należy z poziomu przeglądarki internetowej wejść na stronę internetową:
\begin{verbatim}
	http://127.0.0.1:8000/admin/
\end{verbatim}
oczywiście przy założeniu, że środowisko zostało już uruchomione. Następnie należy się zalogować. Po zalogowaniu ukaże się okno (Rysunek \ref{fig:panelAdmin}).
\begin{figure}[p]
	\centering
	\includegraphics[width=\linewidth]{003}
	\caption{Panel administracyjny}
	\label{fig:panelAdmin}
	\end{figure}
\newline
W oknie tym można zarządzać tabelami oraz grupami użytkowników. 
W przypadku zarządzania użytkownikami można tworzyć, usuwać i zmieniać im uprawnienia. Przykładowo na rysunku \ref{fig:panelUzytkownicy} widać, że utworzony został nowy użytkownik o nazwie buka, ale nie otrzymał on prawa logowania się do panelu administracyjnego (nie została zaznaczona opcja \textit{staff status}). 
\begin{figure}[p]
	\centering
	\includegraphics[width=\linewidth]{005}
	\caption{Panel Django - zmiana ustawień użytkownika}
	\label{fig:panelUzytkownicy}
	\end{figure}
\newline
Dodatkowo panel administracyjny pozwala na pełny dostęp do zawartości bazy - także można nawet usuwać lub dodawać pojedyncze rekordy. Na rysunku \ref{fig:panelPomiary} można zobaczyć fragment zestawienia zawartości tabeli \verb Analyzer_danepomiarowe . 
\begin{figure}[p]
	\centering
	\includegraphics[width=\linewidth]{006}
	\caption{Panel Django - podgląd zawartości tabeli Analyzer\_danepomiarowe }
	\label{fig:panelPomiary}
	\end{figure}
\newline
Z tego zestawienia można wybrać jeden z pomiarów oraz go wyedytować. Edycję wpisu przedstawia rysunek \ref{fig:panelPomiar}.
\begin{figure}[p]
	\centering
	\includegraphics[width=\linewidth]{007}
	\caption{Panel Django - edycja wpisu tabeli Analyzer\_danepomiarowe }
	\label{fig:panelPomiar}
	\end{figure}


\section{Instrukcja obsługi}
\subsection{Instalacja bazy danych}
Baza danych została wdrożona na systemie Linux (przetestowana na dystrybucjach Ubuntu oraz Mint). Aby móc ją uruchomić należy wcześniej zainstalować poniższe pakiety:
\begin{itemize}
	\item coś
\end{itemize}

\subsection{Przygotowanie plików wejściowych}
Ponieważ aplikacja webowa nie pobiera sama danych pogodowych z internetu, należy je załadować z pliku. Aby móc to zrobić trzeba przygotować plik CSV zawierający komplet wymaganych informacji.
W kolejnych wierszach pliku muszą się znaleźć kolejne pomiary. W kolejnych wierszach należy pola oddzielić przecinkami:
	\begin{verbatim}
		nazwa_stacji,data_RRRRMMDD,rodzaj_pomiaru,wartosc_pomiaru
	\end{verbatim}
W przypadku pola \verb data_RRRRMMDD  data musi zostać zapisana w postaci ciągu cyfr nieoddzielonych żadnymi separatorami. Przykładowa zawartość pliku:
	\begin{verbatim}
		Warszawa,20130118,TMIN,-6
		Warszawa,20130119,TMIN,-8
		Warszawa,20130120,TMIN,-6
		Warszawa,20130121,TMIN,-7
	\end{verbatim}

Następnie należy ręcznie zarejestrować rodzaj pomiaru. Zostało to pozostawione stronie administracyjnej, ponieważ wiąże się to bezpośrednio z początkowym wdrażaniem bazy na serwerze. 
W tym celu należy dodać wpisy w odpowiednich tabelach:
\begin{itemize}
	\item \textbf{Analyzer\_jednostka} - dodać opis słowny i ID jednostki związanej z mierzoną cechą
	\item \textbf{Analyzer\_rodzajpomiaru} - dodać opis słowny zgodny z polem \verb rodzaj_pomiaru  z pliku CSV, nadać numer ID oraz odwołać się do klucza wpisanej przed chwilą jednostki
	\item \textbf{Analyzer\_stacja} - dodać nazwę stacji zgodnie z polem \verb nazwa_stacji  z pliku CSV oraz nadać jej numer ID.
\end{itemize}

\subsection{Uruchomienie bazy}
cosikowo





\end{document}
