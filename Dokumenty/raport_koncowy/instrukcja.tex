\subsection{Instalacja bazy danych}
Baza danych została wdrożona na systemie Linux (przetestowana na dystrybucjach Ubuntu oraz Mint). Aby móc ją uruchomić należy wcześniej zainstalować poniższe pakiety:
\begin{itemize}
	\item coś
\end{itemize}

\subsection{Przygotowanie plików wejściowych}
Ponieważ aplikacja webowa nie pobiera sama danych pogodowych z internetu, należy je załadować z pliku. Aby móc to zrobić trzeba przygotować plik CSV zawierający komplet wymaganych informacji.
W kolejnych wierszach pliku muszą się znaleźć kolejne pomiary. W kolejnych wierszach należy pola oddzielić przecinkami:
	\begin{verbatim}
		nazwa_stacji,data_RRRRMMDD,rodzaj_pomiaru,wartosc_pomiaru
	\end{verbatim}
W przypadku pola \verb data_RRRRMMDD  data musi zostać zapisana w postaci ciągu cyfr nieoddzielonych żadnymi separatorami. Przykładowa zawartość pliku:
	\begin{verbatim}
		Warszawa,20130118,TMIN,-6
		Warszawa,20130119,TMIN,-8
		Warszawa,20130120,TMIN,-6
		Warszawa,20130121,TMIN,-7
	\end{verbatim}

Następnie należy ręcznie zarejestrować rodzaj pomiaru. Zostało to pozostawione stronie administracyjnej, ponieważ wiąże się to bezpośrednio z początkowym wdrażaniem bazy na serwerze. 
W tym celu należy dodać wpisy w odpowiednich tabelach:
\begin{itemize}
	\item \textbf{Analyzer\_jednostka} - dodać opis słowny i ID jednostki związanej z mierzoną cechą
	\item \textbf{Analyzer\_rodzajpomiaru} - dodać opis słowny zgodny z polem \verb rodzaj_pomiaru  z pliku CSV, nadać numer ID oraz odwołać się do klucza wpisanej przed chwilą jednostki
	\item \textbf{Analyzer\_stacja} - dodać nazwę stacji zgodnie z polem \verb nazwa_stacji  z pliku CSV oraz nadać jej numer ID.
\end{itemize}

\subsection{Uruchomienie bazy}
cosikowo
