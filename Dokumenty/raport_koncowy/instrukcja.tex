\subsection{Instalacja bazy danych}
Baza danych została wdrożona na systemie Linux (przetestowana na dystrybucjach Ubuntu oraz Mint). Aby móc ją uruchomić należy wcześniej zainstalować poniższe pakiety:
\begin{itemize}
	\item python2.7
	\item python-numpy
	\item python-scipy
	\item python-matplotlib
	\item ipython
	\item ipython-notebook
	\item python-pandas
	\item python-sympy
	\item python-nose
	\item python-statsmodels
	\item python-tk
	\item mysql-connector-python
	\item django-pandas
	\item django-model-utils
	\item scipy
	\item mysql-server
\end{itemize}
Następnie koniecznie należy zaimportować bazę danych poprzez wywołanie poniższej komendy w terminalu w folderze głównym aplikacji (zawierający plik requirements.txt oraz manage.py):
\begin{verbatim}
	python2 manage.py migrate
\end{verbatim}

\subsection{Przygotowanie plików wejściowych}
Ponieważ aplikacja webowa nie pobiera sama danych pogodowych z internetu, należy je załadować z pliku. Aby móc to zrobić trzeba przygotować plik CSV zawierający komplet wymaganych informacji.
W kolejnych wierszach pliku muszą się znaleźć kolejne pomiary. W kolejnych wierszach należy pola oddzielić przecinkami:
	\begin{verbatim}
		nazwa_stacji,data_RRRRMMDD,rodzaj_pomiaru,wartosc_pomiaru
	\end{verbatim}
W przypadku pola \verb data_RRRRMMDD  data musi zostać zapisana w postaci ciągu cyfr nieoddzielonych żadnymi separatorami. Przykładowa zawartość pliku:
	\begin{verbatim}
		Warszawa,20130118,TMIN,-6
		Warszawa,20130119,TMIN,-8
		Warszawa,20130120,TMIN,-6
		Warszawa,20130121,TMIN,-7
	\end{verbatim}

Następnie należy ręcznie zarejestrować rodzaj pomiaru. Zostało to pozostawione stronie administracyjnej, ponieważ wiąże się to bezpośrednio z początkowym wdrażaniem bazy na serwerze. 
W tym celu należy dodać wpisy w odpowiednich tabelach:
\begin{itemize}
	\item \textbf{Analyzer\_jednostka} - dodać opis słowny i ID jednostki związanej z mierzoną cechą
	\item \textbf{Analyzer\_rodzajpomiaru} - dodać opis słowny zgodny z polem \verb rodzaj_pomiaru  z pliku CSV, nadać numer ID oraz odwołać się do klucza wpisanej przed chwilą jednostki
	\item \textbf{Analyzer\_stacja} - dodać nazwę stacji zgodnie z polem \verb nazwa_stacji  z pliku CSV oraz nadać jej numer ID.
\end{itemize}

\subsection{Pierwsze uruchomienie bazy}
Skonfigurowana baza danych wymaga jedynie uruchomienia środowiska, w którym będzie działać aplikacja internetowa. 
W tym celu (przy założeniu, że wszystkie zależności zostały wcześniej zainstalowane) należy stworzyć środowisko:
\begin{verbatim}
	virtualenv --no-site-packages env
	source env/bin/activate
	pip2 install -r requirements.txt --allow-external mysql-connector-python pandas scipy statsmodels django-pandas django-model-utils
\end{verbatim}
Aby uruchomić teraz bazę danych wystarczy:
\begin{verbatim}
	python2 manage.py runserver
\end{verbatim}
I następnie podany w konsoli adres strony skopiować i wkleić w przeglądarce internetowej (domyślnie: \verb http://127.0.0.1:8000/ ).

\subsection{Kolejne uruchomienie bazy}
Przy każdym kolejnym uruchomieniu bazy wystarczy ponownie uruchomić środowisko i serwer:
\begin{verbatim}
	source env/bin/activate
	python2 manage.py runserver
\end{verbatim}
oraz znowu skopiować adres i wkleić go do przeglądarki internetowej. 
