\documentclass[11pt, a4paper, oneside]{mwart}

\def \MYHEADER {Projekt Bazy Danych}
\def \MYTITLE {Analizator danych pogodowych}
\def \MYPDFTITLE {term_1}
\def \MYPDFDATE {\today}
\def \MYAUTHOR {\emph{Autorzy:}\\Aleksandra~\textsc{Grzelak}\\Dorian~\textsc{Janiak}\\Marcin~\textsc{Ochman}}
\def \MYPDFAUTHOR {air}
\def \MYLECTURER {\emph{Prowadzący:} \\dr~in\.z.~G.~\textsc{Myzk}}
\def \MYLSTSETLANGUAGE {matlab}
\def \MYLSTSETFRAME {lines}
\def \MYLSTSETKEYWORDS {} 

\input{_cfg.tex}
\begin{document}
\input{_str_tytulowa.tex}

\newpage
\thispagestyle{empty}

\tableofcontents
\newpage


\section{Cel projektu}

Zadanie polega na stworzeniu aplikacji umożlwiającej analizę i prognozę pogody w oparciu o historię pomiarów. Do tego celu należy zaimplementować: 
\begin{itemize}
  \item  bazę danych, przechowującą pomiary,
  \item algorytm prognozy,
  \item aplikację webową dostarczającą interfejs użytkownika.
\end{itemize}

%\section{Założenia projektowe}

\section{Baza danych}
Bazę danych zaprojektowano w języku \textsc{mySQL}.

Tabele odpowiadają stacjom pomiarowym, rodzajom pomiaru, danym pomiarowym. Na rysunkach \ref{fig:diagram_eer}--\ref{fig:diagram_uml} zostały przedstawione diagramy \textsc{erd} oraz \textsc{uml} prezentujące budowę aplikacji.

\begin{figure}[htbp]
  \centering
  \includegraphics[width=\textwidth]{./diagramEER}
  \caption{Podgląd tabel oraz powiązania pomiędzy nimi w aplikacji.}
  \label{fig:diagram_eer}
\end{figure}

\begin{figure}[htbp]
  \centering
  \includegraphics[width=\textwidth]{./uml}
  \caption{Graficzne przedstawienie klas używanych w aplikacji.}
  \label{fig:diagram_uml}
\end{figure}

\section{Algorytm}

Algorytm prognozy bedzie wykorzystywał różne funkcje: średnią ruchomą, wariancję, korelacje i autokorelacje, różniczkowanie.

\end{document}
