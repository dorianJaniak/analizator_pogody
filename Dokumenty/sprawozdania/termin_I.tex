\documentclass[11pt, a4paper, oneside]{mwart}

\def \MYHEADER {Projekt Bazy Danych}
\def \MYTITLE {Analizator danych pogodowych}
\def \MYPDFTITLE {term_1}
\def \MYPDFDATE {\today}
\def \MYAUTHOR {\emph{Autorzy:}\\Aleksandra~\textsc{Grzelak}\\Dorian~\textsc{Janiak}\\Marcin~\textsc{Ochman}}
\def \MYPDFAUTHOR {air}
\def \MYLECTURER {\emph{Prowadzący:} \\dr~in\.z.~G.~\textsc{Myzk}}
\def \MYLSTSETLANGUAGE {matlab}
\def \MYLSTSETFRAME {lines}
\def \MYLSTSETKEYWORDS {} 

\input{_cfg.tex}
\begin{document}
\input{_str_tytulowa.tex}

\newpage
\thispagestyle{empty}

\tableofcontents
\newpage


\section{Cel projektu}

Zadanie polega na stworzeniu aplikacji umożlwiającej analizę i prognozę pogody w oparciu o historię pomiarów. Do tego celu należy zaimplementować: 
\begin{itemize}
  \item  bazę danych, przechowującą pomiary,
  \item algorytm prognozy,
  \item aplikację webową dostarczającą interfejs użytkownika.
\end{itemize}

%\section{Założenia projektowe}

\section{Baza danych}
Bazę danych zaprojektowano w języku \textsc{mySQL}.

Tabele odpowiadają stacjom pomiarowym, rodzajom pomiaru, danym pomiarowym. Na rysunkach \ref{fig:diagram_eer}--\ref{fig:diagram_uml} zostały przedstawione diagramy \textsc{erd} oraz \textsc{uml} prezentujące budowę aplikacji.

\begin{figure}[htbp]
  \centering
  \includegraphics[width=\textwidth]{./diagramEER}
  \caption{Podgląd tabel oraz powiązania pomiędzy nimi w aplikacji.}
  \label{fig:diagram_eer}
\end{figure}

\section{Algorytm oraz klasy }

Algorytm prognozy będzie wykorzystywał różne funkcje: średnią ruchomą, wariancję, korelacje i autokorelacje, różniczkowanie.

\begin{figure}[htbp]
  \centering
  \includegraphics[width=\textwidth]{./uml}
  \caption{Graficzne przedstawienie klas używanych w aplikacji.}
  \label{fig:diagram_uml}
\end{figure}


\section{Opis poszczególych tabel}
Główną tabelą jest \textit{Analyzer\_danepomiarowe}. Służy ona do przechowywania wszelkich pomiarów, z których będzie korzystać nasza aplikacja. Znajdują się w niej następujące pola:
\begin{itemize}
	\item \textit{id} - identyfikator pomiaru, klucz główny tabeli
	\item \textit{wartość} - wartość pomiaru
	\item \textit{rodzaj\_pomiaru\_id} - identyfikator rodzaju pomiaru, klucz obcy odwołujący się do tabeli \textit{Analyzer\_rodzajpomiaru}
	\item \textit{stacja\_id} - identyfikator stacji badawczej, klucz obcy odwołujący się do tabeli \textit{Analyzer\_stacja}
	\item \textit{data} - data przeprowadzenia pomiaru
\end{itemize}

	Baza danych będzie również zawierać trzy inne tabele:
\begin{itemize}
	\item \textit{Analyzer\_jednostka} - przechowuje informacje o jednostce, zawiera identyfikator jednostki oraz jej nazwę
	\item \textit{Analyzer\_rodzajpomiaru} - przechowuje informacje o możliwych rodzajach pomiaru, zawiera identyfikator rodzaju pomiaru, nazwę pomiaru oraz jednostkę - klucz obcy odwołujący się do tabeli \textit{Analyzer\_jednostka}
	\item{Analyzer\_stacja} - przechowuje informacje o stacjach tj. identyfikator stacji o nazwę tej stacji
\end{itemize}

\end{document}
